\documentclass{beamer}
\usefonttheme{serif}

\usepackage{amsmath,amsthm,amssymb,amsfonts,amscd,mathrsfs,amsxtra,multirow,kotex,mathtools,gensymb,textcomp,lipsum,tikz,verbatim,color,soul,courier,mdframed,xcolor}
\usepackage[normalem]{ulem}
\usetikzlibrary{calc,matrix,arrows,chains,positioning,scopes}
\usepackage{pdfpages}

\theoremstyle{plain}
\newtheorem{thm}{Theorem}[section]
\newtheorem{prop}[thm]{Proposition}

\theoremstyle{definition}
\newtheorem{defn}[thm]{Definition}
\newtheorem{exmp}[thm]{Example}
\newtheorem{excs}[thm]{Exercise}
\newtheorem{rem}[thm]{Remark}
\newtheorem{cor}[thm]{Corollary}
\newtheorem{lem}[thm]{Lemma}
\newtheorem{prob}[thm]{Problem}
\newtheorem{nota}[thm]{Notation}

\newcommand \tr[1]{\textcolor{red}{#1}}
  
\title[]{SE328:Topology}

\author[]{Hyosang Kang\inst{1}}

\institute[]{\inst{1}Division of Mathematics\\ School of Interdisciplinary Studies\\ DGIST}

\date[]{Week 01}

\begin{document}

\begin{frame}
\titlepage
\end{frame}

\begin{frame}
\begin{defn}
	Given a statement of the form $P\Rightarrow Q$,
	its contrapositive is $\sim Q\Rightarrow P$,
	and its converse is $Q\Rightarrow P$,
	where $\sim P$ is the negation of $P$.
\end{defn}
\begin{exmp}
	\begin{enumerate}
		\item $A\subset B$ and $A\subset C$ 
		$\Leftrightarrow$ $A\subset(B\cap C)$
		\item $A - (A - B) = B$
		\item $(A\cap B)\cup (A-B) = A$
	\end{enumerate}
\end{exmp}
\begin{exmp}
	\begin{enumerate}
		\item Write the contrapositive and 
		converse of $x<0\Rightarrow x^2-x>0$
		\item Write the negation of
		$\forall a\in A, a^2\in B$
	\end{enumerate}
\end{exmp}
\end{frame}

\begin{frame}
\begin{defn}
	Given sets $A$ and $B$, the cartesian product
	$A\times B$ is the set 
	$$A\times B = \{(a,b)\mid a\in A,\,b\in B\}$$
\end{defn}
\begin{exmp}
	Determine whether each of the following sets is a
	cartesian product of two subsets of $\mathbb R$.
	\begin{enumerate}
		\item $\{(x,y)\mid x\in\mathbb Z\}$
		\item $\{(x,y)\mid y>x\}$ 
	\end{enumerate}
\end{exmp}
\end{frame}

\begin{frame}
\begin{defn}
	A function $f$ is a subset of the cartesian 
	product $A\times B$ of two sets, 
	with the property that
	each elemetn in $C$ appears as the first
	coordinate of at most one ordered pair.
	In other words, 
		$$(a,b), (a,b')\in f\Rightarrow b=b'$$
\end{defn}
\begin{defn}
	Given a function $f:A\to B$ and a subsets
	$A_0\subset A$, 
	the restriction of $f$ to $A_0$ is
	$$\{(a,f(a))\mid a\in A_0\}$$
\end{defn}
\end{frame}

\begin{frame}
\begin{defn}
	A function $f:A\to B$ is injective if 
	$$\forall a\in A, f(a)=f(a')\Rightarrow a=a'$$
	and surjective if
	$$\forall b\in B, \exists a\in A f(a) = b$$
\end{defn}
\begin{exmp}
	Let $f:A\to B$ and $g:B\to C$ be functions and
	$A_0\subset A$, $B_0, B_1\subset B$.
	\begin{enumerate}
		\item $A_0\subset f^{-1}(f(A_0))$.
		\item $B_0\subset B_1$ $\Rightarrow$ 
		$f^{-1}(B_0)\subset f^{-1}(B_1)$.
		\item If $C_0\subset C$, then
		$$(g\circ f)^{-1}(C_0) 
			= f^{-1}(g^{-1}(C_0))$$
	\end{enumerate}
\end{exmp}
\end{frame}

\begin{frame}
\begin{defn}
	A equivalence relation $\sim$ on a set $A$ 
	is a subset $\subset A\times A$ such that
	\begin{enumerate}
		\item $\forall x\in A, (x,x)\in \sim$
		\item $(x,y)\in C\Rightarrow (y,x)\in \sim$
		\item $(x,y), (y,z)\in \sim 
			\Rightarrow (x,z)\in \sim$
	\end{enumerate}
	We denote $x\sim y$ if $(x,y)\in \sim$.
\end{defn}
\begin{defn}
	The equivalence class of $x\in A$ is the set
	$$[x] = \{y\mid y\sim x\}$$
	The collection of all equivalence classes 
	for $\sim$ becomes a partition of $A$, 
	i.e. the collection of
	disjoint nonempty subsets of $A$.  
\end{defn}
\end{frame}

\begin{frame}
\begin{defn}
	An order relation $<$ on a set $A$ 
	is a subset of $A\times A$ such that
	\begin{enumerate}
		\item $x,y\in A, x\neq y \Rightarrow
			(x,y)\in < \textrm{ or } (y,x)\in <$
		\item $\nexists x\in A, (x,x)\in <$
		\item $(x,y)\in <, (y,z)\in < \Rightarrow
			(x,z)\in <$.
	\end{enumerate} 
	We denote $x<y$ if $(x,y)\in <$.
\end{defn}

\begin{defn}
	If $<$ is an order relation on a set $A$,
	and if $a<b$, an open interval $(a,b)$ 
	is a subset defined by
		$$(a,b) = \{x\in A\mid a<x<b\}$$
	If $(a,b) = \emptyset$, then $a$ is called the
	immediate predecessor of $b$, and $b$ is called
	the immediate successor of $a$.
\end{defn}
\end{frame}

\begin{frame}
\begin{defn}
	Suppose that $A, B$ are two sets with order
	relations $<_A, <_B$. The dictionary order
	relation $<$ on $A\times B$ is defined by
		$$a_1\times b_1 < a_2\times b_2$$
	if $a_1<_Aa_2$, or if $a_1=a_2$ and $b_1<_Bb_2$.
\end{defn}
\begin{defn}
	An ordered set $A$ is said to have the 
	least upper bound property if every nonempty
	subset $A_0\subset A$ that is bounded above has
	a least upper bound. The greatest lower bound
	property is similary defined.
\end{defn}
\end{frame}

\begin{frame}
\begin{exmp}
	\begin{enumerate}
		\item Let $f:A\to B$ is a surjective function.
		Define $a_0\sim a_1$ if $f(a_0)=f(a_1)$.
		Show that $\sim$ is an equivalence relation.
		\item If an ordered set $A$ has the least
		upper bound property, then it has the
		greatest lower bound property.
		\item Showt that $[0,1] 
		= \{x\mid0\le x\le 1\}$ has the least upper
		bound property.
		\item Does $[0,1]\times[0,1]$ in the dictionary
		order relation have the least upper bound
		property?
	\end{enumerate}
\end{exmp}	
\end{frame}

\begin{frame}
\begin{defn}
	The set of real numbers, denoted by $\mathbb R$,
	is a set with two operations $+$(addition), 
	$\cdot$(multiplication), 
	and an order relation $<$.
	It contains two special elements,
	$1$(one) and $0$(zero).
	All elementary algebraic properties hold 
	including the following statements.
	\begin{enumerate}
		\item $x<y\Rightarrow x+z<y+z$
		\item $x<y, 0<z \Rightarrow 
			x\cdot z<y\cdot z$
		\item $<$ has the least upper bound property
		\item $x<y\Rightarrow\exists z\in \mathbb R
			x<z<y$
	\end{enumerate}
\end{defn}
\end{frame}

\begin{frame}
\begin{defn}
	The subset $A\subset\mathbb R$ is called 
	inductive if it contains $1$ and 
	if $x\in A$ then $x+1\in A$.
	The set of all positive integers, denoted by 
	$\mathbb Z_+$ is the smallest among all 
	inductive subsets.
\end{defn}
\begin{thm}
	Every nonempty subset of $\mathbb Z_+$ has 
	a smallest element.
\end{thm}
\begin{thm}
	Let $A$ be a set of positive integers.
	For each positive integer $n\in\mathbb Z_+$,
	$S_n\subset A$ $\Rightarrow$ $n\in A$.
	If this is true for all $n$, 
	then $A=\mathbb Z_+$.
\end{thm}
\end{frame}

\begin{frame}
\begin{defn}
	Let $m$ be a positive integer.
	An $m$-tuple of elements of $X$ is a function
	$\mathbf x:\{1,\cdots,m\}\to X$.
	The $\omega$-tuple of elements of $X$ is
	a function $\mathbf x:\mathbb Z_+\to X$.
\end{defn}
\begin{defn}
	Let $A_1,A_2,\cdots$ be a family of sets,
	indexed by $\mathbb Z_+$.
	The cartesian product of $A_i$, denoted by 
	$\displaystyle\prod_{i\in\mathbb Z_+}A_i$,
	is the set of all $\omega$-tuples of elements of 
	$\displaystyle\bigcup_{i\in\mathbb Z_+}A_i$ 
	such that $x_i\in A_i$.
	If $A_i=X$ for all $i$, then the cartesian
	product is denoted by $X^\omega$.
\end{defn}
\begin{exmp}
Find a bijective map $f:X^\omega\times X^\omega\to
X^\omega$
\end{exmp}
\end{frame}

\begin{frame}
\begin{defn}
	A set is called finite if there 
	is a bijection between the set and $S_n$ 
	for some positive integer $n$.
	A set is called infinite if it is not finite.
	It is called countably infinite if there is
	a bijection between the set and $\mathbb Z_+$.
	A infinite set which is not countable
	is called uncountable.
\end{defn}
\begin{thm}
	Let $A$ be a set. The followings are equivalent:
	\begin{enumerate}
		\item $A$ is countable.
		\item There is a surjective function 
		$f:\mathbb Z_+\to A$.
		\item There is an injective function
		$f:A\to\mathbb Z_+$. 
	\end{enumerate}
\end{thm}
\end{frame}

\begin{frame}
\begin{lem}
	If $A$ is an infinite subset of $\mathbb Z_+$,
	then $A$ is countably infinite.
\end{lem}
\begin{cor}
	\begin{enumerate}
		\item A subset of a countable set is countable.
		\item The set $\mathbb Z_+\times \mathbb Z_+$
		is countably infinite.
		\item A countable union of countable sets
		is countable.
		\item A finite product of countable sets
		is countable.
	\end{enumerate}
\end{cor}
\begin{thm}
	Let $X$ be the two element set $\{0,1\}$.
	Then $X^\omega$ is uncountable.
\end{thm}
\end{frame}

\begin{frame}
\begin{defn}
	Two sets $A$ and $B$ have the same cardinality
	if there is a bijection between $A$ and $B$.
\end{defn}
\begin{exmp}
	\begin{enumerate}
		\item Show that if $B\subset A$ and 
		there is a injection $f:A\to B$,
		then $A$ and $B$ have the same cardinality.
		\item If there are injection $f:A\to C$
		and $g:C\to A$, then $A$ and $C$ have the
		same cardinality.
		\item Let $X$ be the two element set $\{0,1\}$,
		and $\mathcal B$ be the set of all countable 
		subsets of $X^\omega$. Then $X^\omega$
		and $\mathcal B$ have the same cardinality.
	\end{enumerate}
\end{exmp}
\end{frame}

\begin{frame}
\begin{thm}
	Let $A$ be a set.
	The followings are equivalent.
	\begin{enumerate}
		\item There is an injective function 
		$f:\mathbb Z_+\to A$.
		\item There is a bijection between $A$
		and a proper subset of $A$.
		\item $A$ is infinite.
	\end{enumerate}
\end{thm}
\begin{block}{Axiom of choice}
	Given a collection $\mathcal A$ of 
	disjoint nonempty sets, there exists a set
	$C$ consisting of exactly one element from
	each element of $\mathcal A$.
\end{block}
\begin{exmp}
	Define an injective map 
	$f:\mathbb Z_+\to X^\omega$ where $X=\{0,1\}$
	with (or without) using axiom of choice.
\end{exmp}
\end{frame}

\begin{frame}
\begin{defn}
	A set $A$ with an order relation $<$ is called
	well-ordered if every nonempty subset of $A$
	has a smallest element.		
\end{defn}
\begin{defn}
	Two ordered sets $A$ and $B$ have the same
	order type if there is a bijection between 
	$A$ and $B$ preserving order relations.
\end{defn}
\begin{thm}
	Every nonempty finite ordered set has 
	the order type of a section $\{1,\cdots,n\}$
	of $\mathbb Z_+$
\end{thm}
\begin{exmp}
	\begin{enumerate}
		\item Show that $\{1,2\}\times\mathbb Z_+$
		in dictionary order is well-ordered.
		\item Do $\{1,2\}\times\mathbb Z_+$ and
		$\mathbb Z_+\times\{1,2\}$ have the same 
		order type?
	\end{enumerate}
\end{exmp}
\end{frame}
\end{document}